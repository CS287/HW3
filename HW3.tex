\documentclass[11pt]{article}
\usepackage{writeup/common}
\usepackage{url}
\usepackage{listings}

\title{CS 287 \\ Assignment 2: Neural Language Modeling }
\date{}
\begin{document}

\maketitle{}

\begin{center}
  \textbf{Due:} Monday, March 7th, 11:59 pm 
\end{center}


This assignment focuses on language modeling. 
 we will be implementing several different 
language models. \cite{DBLP:journals/jmlr/BengioDVJ03}

%  influential neural network model
% for part-of-speech tagging, based on the paper \textit{NLP (Almost)
%   from Scratch} \citep{collobert2011natural}. The goal of this
% homework assignment is to replicate the central model from this
% work. At the end of this assignment, you will have created your own
% state-of-the-art part-of-speech tagger, and built a framework that can
% be used easily for many other NLP tasks.


As you complete this assignment, we ask that you submit your results
on the test data to the Kaggle competition website at
\url{https://inclass.kaggle.com/c/cs287-hw3} and that you compile your
experiences in a write-up based on the template at
\url{https://github.com/cs287/hw_template}.

\section{Data and Preprocessing}

\subsection{Data}

The main data for this task is in the \texttt{data/} directory. Here
is the first few sentences of validation. Each line of the file
contains one tokenized sentence.

\lstset{breaklines=true}
\begin{lstlisting}
> head data/valid.txt 
 consumers may want to move their telephones a little closer to the tv set 
 <unk> <unk> watching abc 's monday night football can now vote during <unk> for the greatest play in N years from among four or five <unk> <unk> 
 two weeks ago viewers of several nbc <unk> consumer segments started calling a N number for advice on various <unk> issues 
 and the new syndicated reality show hard copy records viewers ' opinions for possible airing on the next day 's show 
 interactive telephone technology has taken a new leap in <unk> and television programmers are racing to exploit the possibilities 
 eventually viewers may grow <unk> with the technology and <unk> the cost 
 but right now programmers are figuring that viewers who are busy dialing up a range of services may put down their <unk> control <unk> and stay <unk> 

\end{lstlisting}

In addition to we have also included a set of data with a smaller vocabulary (which 
will be much faster to train on). This data is in \texttt{train.smallvocab.txt}. 
These sentences are identical, but have many more word types replaced with UNK.

We also include dictionary files, 

\begin{verbatim}
> head  data/words.1000.dict 
\end{verbatim}

This is a mapping from each word in the Penn Treebank to
a unique ID. It is important that you use this mapping since it 
is used for the Kaggle competition data. 

\subsection{Preprocessing}

For this assignment you will write your own preprocessing code
in \texttt{preprocessing.py}. This code should transform the
data representations given above into a format for multiclass
prediction. In particular you will be responsible for,

\begin{itemize}
\item Mapping the words in the data set to indexes.
\item Pre-construct the n-gram windows before word.
\end{itemize}

Your preprocessing code should include an argument for the size of 
the n-gram that is used.

\textbf{Important note:} the boundary cases for language modeling are
a bit tricky. For each sentence, you must predict the end-of-sentence
token represented as "\texttt{</s>}". Conversely you do not predict the
start-of-sentence token "<s>" but you do condition on it when
predicting words at the beginning of the sentence. Therefore the
number of words predicted is size of words in the file + number of
sentences (one \texttt{</s>} for each sentence).


For the test set, we have blanked out one word in every sentence,
instead writing \texttt{\_\#\_} where \# is the example number .  The
evaluation, discussed below, will be to give the full distribution
over this missing word. Therefore when constructing your \texttt{test\_input}
you should only output the context for these examples.

\begin{quote}
  train\_input, train\_output,
  valid\_input,
  valid\_output
  test\_input, nwords
\end{quote}

\section{Code Setup}

Write your main code in \texttt{HW3.lua}. For this assignment part,
you can (and should) use the \texttt{nn} library in addition to the
standard Torch library. 

\subsection{Prediction and Evaluation}

Be sure you are able to read and output the text data and that you
understand the format.  Also be sure you are able output your
classification results on the test data as a text file. Check that you
can upload these to the Kaggle. 

The Kaggle submission for this assignment takes on a more complicated form. 
For each position in the test data, we would like you to output a complete 
distribution over possible words.

For instance the first blank in test is 

\begin{verbatim}
 no it was n't \_1\_ monday 
\end{verbatim}

In your output file you should give the full distribution $p(w_5 | w_1, \ldots, w_4)$.

\begin{verbatim}
ID,Class1,Class2,Class3,...
1,0.2,0.001,0.0012,...
\end{verbatim}

Here Class1 refers to the first word in \texttt{words.dict}, Class2 to the second, etc.

\subsection{Hyperparameters}

Several of the models described have explicit hyperparameters that you will 
need to tune. It is your responsibility to cleanly separate these out from 
the models themselves and expose as command-line options. This makes it much 
easier to run experiments and to utilize experimental scripts. 

\subsection{Logging and Reporting}

As part of the write-up, you will need to report on the training and
predictive accuracy of your models. To make this possible, your code
should report on various metrics of the model both at training and
test time. We will leave it up to you on which metrics to log, but we
recommend reporting training speed, training set NLL, training set
predictive accuracy, and validation predictive accuracy. It is your
responsibility to convince us that the model is correctly training.

\section{Models}

For this assignment you will implement the three models. We will warm
up with a simple count-based language model, then we will add
smoothing, and finally we will move on to the full neural network
model from \citet{bengio} and finally implement HSM training.

\subsection{Count-Based Language Model}

To start, modify your code from HW1 to implement a simple of
n-gram multinomial estimate of $p(w_i | w_{i-n}, \ldots w_{i-1})$. 
Use three different strategies to estimate this distribution,

\begin{enumerate}
\item Simple maximum likelihood estimation. 
\item Estimation with Laplace smoothing $\alpha$.
\item Estimation with Kneser-Ney smoothing \cite{}. 
\end{enumerate}

Use this distribution to calculate the perplexity of the development
set with n-grams 2, 3, and 4.


The formula for Kneser-Ney Smoothing is: 

\subsection{Neural Network Language Models (NNLMs)}

To compare, we will also implement a neural network language model for
this problem. Unfortunately when using a CPU it is too inefficient to
train on this full data set. The issue comes from the partition
function, which requires $O(|\mcV|)$ time to compute each step. To
avoid this issue, we have provided a reduced vocabulary data set
\texttt{data/train.1000.txt} to start with. In the next section we 
will get to the full data set.

Note the following correspondences,

\begin{center}
  \begin{tabular}{ll}
    \toprule
    Math & Torch \\
    \midrule 
    sparse $\boldx \boldW$ & nn.LookupTable \\ 
    $\log \mathrm{softmax}$ & nn.LogSoftMax \\ 
    $\tanh $ & nn.TanH \\ 
    $L_{cross-entropy}()$ & nn.ClassNLL \\ 
    dense $\boldx \boldW + \boldb$ & nn.Linear \\ 
    \bottomrule
  \end{tabular}
\end{center}
The documentation (\url{https://github.com/torch/nn}) is quite good
for all of these models (although it is possible the notation may be
slightly different than class).

The training and evaluation code for the model takes a similar form to
HW2. Similar tips apply here .
 
 \begin{itemize}
 \item Train using minibatches (we used size 32). This will significantly speed up training. 
 \item Keep track of the average loss as you train. 
 \item Run on the development set each iteration to see the progress of the model.
 \item You should not have to run for more than 20 epochs on this data. If you are seeing large 
   changes beyond that, you may have an issue. 
 \item One simple way of regularizing language models is to
   renormalize the word embeddings to have a max $\ell2$ norm of 1
   after each epoch. We recommend this technique instead of other
   types of regularization.
 \end{itemize}

\subsection{Neural Network Model with Noise Constrastive Estimation}

Noise constrastive estimation.

\subsection{Word Embeddings}

Once you have trained your model, you can examine the word embeddings that the model 
produces by exporting the parameters of your \texttt{nn.LookupTable}. Write out these 
parameters to HDF5 and load them into back into Python. There are several interesting 
experiments that are often run on these embeddings. We ask that you demonstrate the 
correctness of your approach with some of these methods.

\begin{itemize}
\item For a subset of words compute the k-nearest neighbors under dot-product and
  cosine similarity. What does this show? How do the metrics differ?
\item Graph a subset of the word vectors by projecting to 2-dimensions.  The simplest method is to use PCA, but there are a number of other techniques. Scikit-Learn has a number of 
  methods for doing this which you can explore through the tutorial at \url{http://scikit-learn.org/stable/auto_examples/manifold/plot_lle_digits.html}.
\end{itemize}

\subsection{Additional Experiments}

Once these models are constructed, you should also report on
additional experiments on these data sets. We will leave this aspect
open-ended, but suggestion include:

\begin{itemize}
\item Including additional features in the model. For instance, capitalization or suffixes.
\item Experimenting with different architectures (Replicate the different setups in \cite{}).
\item Modern language modeling experiments are often run on a GPU, an
  architecture particularly well-suited for computing large partition
  functions. If you have access to a NVidia GPU, we encourage you to
  tree these experiments. Follow the instructions on the Torch website for running on this 
  system. It is usually as easy as calling :cuda on your model.
\end{itemize}

\section{Report and Submission}

For your write-up, follow the report template at
\url{https://github.com/cs287/hw_template}. Be sure to include a link
to your code, Kaggle ID, and reports on your results.

In addition to submitting your Kaggle results, we also expect you to report on your 
experimental process. This should include data tables, graphs and discussion of any 
issues that you may run into. 

\bibliographystyle{apalike} 
\bibliography{HW3}

\end{document}
